\documentclass{article}

%-------------------------------------------------

\usepackage{fullpage}
\usepackage{setspace}

\usepackage{amsmath}
\usepackage{amsfonts}
\usepackage{amssymb}

\usepackage{color}

%------------------------

%-------------------------------------------------
\begin{document}

\title{
Upon constraining the spatial distribution of transient gravitational-wave sources
}

\author{
Reed Essick
}

\maketitle

%------------------------

\doublespace

We begin by asserting that graviational waves (GWs) are interesting and therefore their sources are interesting.
Furthermore, we assert that the spacetime distribution of sources is also interesting and set out to measure that distribution using transient GW signals.
A bit of nomenclature

\vspace{0.5cm}
\begin{tabular}{p{3cm}p{12cm}}
    $R(m_1, m_2, z, \Omega)$ & the rate of signals from a direction in the sky ($\Omega$) at redshift $z$ with masses $m_1$, $m_2$, etc, measured with respect to detector-frame time. \\
    $dV_c/d\Omega dz$ & the differntial sensitive co-moving volume observed by our detectors as a function of time, position on the sky, and $z$, which depends on the detector sensitivity. \\
    $p(\mathrm{noise})$ & the probability of observing a particular noise realization. This is usually Gaussian in a frequency series or a non-central $\chi^2$-distribution with 2 degrees of freedom for the signal-to-noise ratio. \\
    $\mathcal{F}(m_1,m_2,z,\Omega;t) $ & the probability of detecting a signal with these parameters at this time (related to probabilities of specific noise realizations and the associated signal-to-noise ratios); in most cases this may be treated approximately as a step function.
\end{tabular}
We note that 
\begin{equation}
    \mathcal{F} = \int \mathcal{D}\mathrm{noise}\, \left(p(\mathrm{noise}) R\frac{dV_c}{d\Omega dz}\right) \Theta\left(\mathrm{noise}+h(m_1, m_2, \Omega, z)\geq \mathrm{thr}\right)
\end{equation}
where $\Theta$ is the heaviside function and $h$ is the modeled strain in the detector associated with $m_1$, $m_2$, $\Omega$, and $z$.

\vspace{0.5cm}
\noindent
We also assume that reasonable priors for our Bayesian inference on a per-event basis follow the modeled astrophysical distribution so that
\begin{equation}
    \frac{p(m_1, m_2, z, \Omega|t,R)}{p(m_1^\prime, m_2^\prime, z^\prime, \Omega^\prime|t^\prime,R)} = \frac{R(m_1, m_2, z, \Omega) \left.(dV_c/d\Omega dz)\right|_{\Omega, z}}{R(m_1^\prime, m_2^\prime, z^\prime, \Omega^\prime) \left.(dV_c/d\Omega dz)\right|_{\Omega^\prime, z^\prime}}
\end{equation}
This means that the expected number of signals with masses $m_1$, $m_2$, etc. coming from $\Omega$ at $z$ coincident with $t$ will be something like 
\begin{equation}
    N_\mathrm{astro} = \left[ R(m_1, m_2, \Omega, z) \frac{dV_c}{d\Omega dz} \Delta\Omega \Delta z \Delta m_1 \Delta m_2 \right] \Delta t
\end{equation}
and we expect to detect 
\begin{equation}
    N_\mathrm{detect} = N_\mathrm{astro} \mathcal{F}(m_1, m_2, z, \Omega; t)
\end{equation}

Now, we consider rates that are low enough so that only one type of signal ($m_1$, $m_2$, $z$, $\Omega$) is detectable at a time and further model each type of signal as an independent Poisson process.
We then marginalize over all possible signal parameters, which we approximate by a set of discrete bins indexed by $k$, and noise realizations, which we approximate by an indepdendent set of discrete bins indexed by $\alpha$.
For each combination of $k$ and $\alpha$, we expect the rate of detectable signals in the detectors to be $\mathrm{rate} = R_k dV_k p(n_\alpha) \Theta_{k,\alpha}$, where we've abbreviated some of the terms from above.
This is because there will be an overall rate of $R_k dV_k$ astrophysical events in our detectors, and the relative frequency with which each of these events will have noise realization $n_\alpha$ is $p(n_\alpha)$.
Finally, if the signal and noise are not above threshold, then the rate of signals above threshold vanishes and this is captured by $\Theta_{k,\alpha}$.
Thus, we expect
\begin{align}
    p(\mathrm{data}=d|\mathrm{signal}; R, t)p(\mathrm{signal}|R, t) & = \left. \sum_{k,\alpha} \right[ p(N_{k,\alpha}=1|R_k dV_k p(n_\alpha)\Theta_{k,\alpha} \Delta t) \prod\limits_{j \neq k, \beta \neq \alpha} p(N_{j,\beta}=0|R_j dV_j p(n_\beta)\Theta_{j,\beta} \Delta t) \nonumber \\
                                                                  & \quad\quad\quad\quad \left. \times p\left(d\left|n_\alpha, m_1^{(k)}, m_2^{(k)}, \Omega^{(k)}, z^{(k)}\right)\right. \right] \nonumber \\
                                                                  & = \left. \sum_{k,\alpha} \right[ \left(R_k dV_k p(n_\alpha)\Theta_{k,\alpha} \Delta t\right) e^{-R_k dV_k p(n_\alpha)\Theta_{k,\alpha} \Delta t} \prod\limits_{j \neq k, \beta \neq \alpha} e^{-R_j dV_j p(n_\beta)\Theta_{j,\beta} \Delta t} \nonumber \\
                                                                  & \quad\quad\quad\quad \left. \times \delta\left(d - n_\alpha+h^{(k)}\right) \right] \nonumber \\
                                                                  & = \left[\sum_{k} \left(R_k dV_k p\left(n=d-h^{(k)}\right) \Delta t\right) \right] e^{-\sum\limits_{j, \beta} R_j dV_j p(n_\beta)\Theta_{j,\beta} \Delta t} \nonumber \\
                                                                  & = \Delta t \left[ \int dm_1 dm_2 dz d\Omega \left(R \frac{dV_c}{d\Omega dz} \right) p\left(d|m_1, m_2, \Omega, z\right) \right] e^{-\left[\int dm_1 dm_2 dz d\Omega \left(R \frac{dV_c}{d\Omega dz} \mathcal{F}\right)\right] \Delta t}
\end{align}
where in the second-to-last line we use the fact that $(d\geq\mathrm{thr})$ to drop the factor of $\Theta_{k,\alpha}$ because there is a detectable signal in the data, and in the last line we've assumed the standard notation $p(d|m_1,m_2,\Omega,z)=p(n=d-h(m_1,m_2,\Omega,z)$.
Similarly, we obtain
\begin{align}
    p(\mathrm{data}=d|\mathrm{noise}; R, t)p(\mathrm{noise}|R, t) & = p(n=d) \prod\limits_{j,\beta} p(N_{j,\beta}=0|R_jdV_j p_\beta(\mathrm{noise})\Theta_{j,\beta} \Delta t) \nonumber \\
                                                                & = p(d|\mathrm{noise}) e^{-\left[\int dm_1 dm_2 dz d\Omega \left(R \frac{dV_c}{d\Omega dz} \mathcal{F}\right)\right] \Delta t} 
\end{align}
where I've again switched to more standard notation in the last line: $p(d|\mathrm{noise}) = p(n=d)$.
These define the relevant likelihoods for each small time segment.
We assume many time segments are independent and then stack everything together.
This give an overall joint likelihood of observing the set of detections we actually observed, conditioned on a Rate distribution: $R(m_1, m_2, z, \Omega)$.
Following Messenger+Veitch (2013), we obtain
\begin{align}
    p(\mathrm{data}|R) = & \prod\limits_{i|d_i\geq\mathrm{thr}} \left[ p(d_i|\mathrm{noise}) + \int dm_1 dm_2 dz d\Omega \left( R\frac{dV_c}{d\Omega dz} \Delta t \right) p(d_i|m_1, m_2, z, \Omega) \right] e^{-\int dm_1 dm_2 dz d\Omega R\frac{dV_c}{d\Omega dz} \mathcal{F}_i \Delta t} \nonumber \\
                         & \quad\quad\quad\quad \times \prod\limits_{j|d_j<\mathrm{thr}} p(d_i<\mathrm{thr}|\mathrm{noise}) e^{-\int dm_1 dm_2 dz d\Omega R\frac{dV_c}{d\Omega dz} \mathcal{F}_j \Delta t} 
\end{align}
which we simplify by assuming $p(d_i<\mathrm{thr}|\mathrm{noise}) =1$.
This is equivalent to normalizing the likelihood appropriately to assume we only see signals above our threshold. 
In reality, we should be able to extract more information by lowering the threshold to include marginal events, but the associated benefits are expected to be realtively small with a significant computational overhead.
With our simplifying assumptions, we obtain
\begin{multline}
    \log p(\mathrm{data}|R) = - \int dt dm_1 dm_2 dz d\Omega \left(R \frac{dV_c}{d\Omega dz} \mathcal{F}\right) \\
         + \sum\limits_{i|d_i\geq\mathrm{thr}} \log \left[ \Delta t \int dm_1 dm_2 dz d\Omega \left(R \frac{dV_c}{d\Omega dz} \right) p(d_i|m_1, m_2, \Omega, z) \right] \\
         + \text{terms independent of }R
\end{multline}
This looks something like a Poisson factor for the overall rate along with the evidence for each individual event.
We can use this marginal likelihood for the Rate distribution to make inferences about the rate.
Several approaches for sampling $p(R|\mathrm{data})$ have been proposed
\begin{itemize}
    \item bin the sky into separate voxels and run an MCMC direcly over the rate in each voxel. This can be expensive if there are a lot of voxels. We note that sampling can be done in ``pixel-space'' or in spherical harmonics, although care is needed to guarantee $R(\Omega) \geq 0\ \forall\ \Omega$.
    \item assume the deviations from isotropy are small and attempt to infer the posterior of the deviations analytically (see below).
\end{itemize}
along with several sanity checks and/or back-of-the-envelop estimates
\begin{itemize}
    \item divide the sky into hemispheres and count the posterior weight in each hemisphere as a statistic looking for a dipole. We note that because ground-based detectors' antenna patterns are quadropoles, there manifestly must be equal exposures in any hemisphere and it's complement, meaning a counting experiment is justified without additional weighting by exposure.
    \item compute the 2-point auto-correlation function with stacked posteriors to look for preferred length scales. This may have to deal with complicated ``point-spread functions'' associated with the posterior distributions of each event.
\end{itemize}

%------------------------

\subsection*{Marginal likelihood for $R_\Omega$}

We assume a separation of the rate distribition: $R = p_{m,z}(m_1, m_2, z) R_\Omega(\Omega)$.
We have adopted a separation of variables in which the mass and redshift distribution integrates to one and the units are carried by the angular dependence.
Thus,
\begin{align}
    \mathcal{N} p_\Omega & = \int dm_1 dm_2 dz\, \left(p_{m,z} \frac{dV_c}{d\Omega dz} \right) \\
    \int d\Omega p_\Omega & = 1
\end{align}
so that
\begin{equation}
    \int dm_1 dm_2 dz \left( p_{m,z}\frac{dV_c}{d\Omega dz} \right) p(d_i|m_1, m_2, z, \Omega) = \mathcal{N} p(\Omega|d_i)
\end{equation}
where $p(\Omega|d_i)\propto p(d_i|\Omega)p_\Omega$ is the a posteriori skymap for event $i$.
We find
\begin{align}
    \log p(\mathrm{data}|R) \supset & - \int d\Omega\, R_\Omega \int dt \mathcal{N} p_\Omega + \sum\limits_{i|d_i\geq\mathrm{thr}} \left[ \log\left(\int d\Omega\, R_\Omega p(\Omega|d_i)\right) + \log\left(\Delta t \mathcal{N}\right) \right] 
\end{align}
We note that $R_o = \int d\Omega R_\Omega$ sets the overall rate of signals detectable in our detectors.
This also shows that inference about $R_\Omega$ can be performed with knowledge of the a posteriori skymaps from each detection ($p(\Omega|d_i)$) and overall measures of the exposure as a function of $\Omega$ integrated over time ($\int dt \mathcal{N} p_\Omega$).
If we assume an isotropic distribution ($R_\Omega = \text{constant}$), we obtain the expected inference for the overall rate of signals $R_o$.

We also note that, because $R_\Omega$ only appears within inner products (integrals over $\Omega$), we can replace it by sums over spherical harmonics such that $R_\Omega = \sum\limits_{l,m} a_{lm} Y_{lm}(\Omega)$.
Thus, 
\begin{equation}
    \log p(\mathrm{data}|a_{lm}) \supset - \sum\limits_{l,m} a_{lm} E_{lm} + \sum\limits_{i|d_i\geq\mathrm{thr}} \log\left(\sum\limits_{lm} a_{lm} P_{lm}^{(i)} \right)
\end{equation}
where $E_{lm}$ are the spherical harmonic coefficients corresponding to the exposure and $P_{lm}^{(i)}$ are the spherical harmonic coefficients corresponding to each event's posterior distribution.
Special care must be taken to guarantee that $R(\Omega)\geq0\ \forall\ \Omega$, which can place complicated joint restrictions on the $a_{lm}$.

%------------------------

\subsection*{Analytic regression of small (perturbative) anisotropies}

Let us assume that $R_\Omega$ is well described by a simple model $R_o(\Omega)$, perhaps containing only a few spherical harmonics or otherwise describing only large-scale features.
We then consider small perturbations to the simple model so that 
\begin{equation}
    \left. R_\Omega = R_o(\Omega)\left(1+\varepsilon(\Omega)\right) \ \right| \ \varepsilon \ll 1 \ \forall \ \Omega
\end{equation}
This yields
\begin{align}
    \log p(\mathrm{data}|R) \supset & \left( - \int d\Omega R_o \int dt \mathcal{N}p_\Omega + \sum\limits_{i|d_i\geq\mathrm{thr}} \left( \log\left[ \int d\Omega R_o \mathcal{N}p_\Omega \Delta t \right] \right) \right) \nonumber \\
                                    & \quad \quad \quad \quad + \left( \int d\Omega\, \varepsilon R_o \left[ - \int dt \mathcal{N} p_\Omega + \sum\limits_{i|d_i\geq\mathrm{thr}} p(\Omega|d_i) \right] \right) \nonumber \\
                                    & \quad \quad \quad \quad \quad \quad \quad \quad - \left( \frac{1}{2}\int d\Omega d\Omega^\prime \varepsilon(\Omega) \left[ R_o(\Omega) R_o(\Omega^\prime) \sum\limits_{i|d_i\geq\mathrm{thr}} p(\Omega|d_i) p(\Omega^\prime|d_i) \right] \varepsilon(\Omega^\prime) \right) \nonumber \\
                                    & \quad \quad \quad \quad \quad \quad \quad \quad \quad \quad + O\left(\varepsilon^3\right)
\end{align}
We note that, in this approximation, the likelihood is a well-behaved Gaussian in $\varepsilon$ and therefore amenable to analytic priors.
Specifically,
\begin{align}
    0 & \leq \left[\int d\Omega \varepsilon R_o \sum\limits_{i|d_i\geq\mathrm{thr}} p(\Omega|d_i)\right]^2 \nonumber \\
      & = \int d\Omega d\Omega^\prime \varepsilon(\Omega) \varepsilon(\Omega^\prime) \left[ R_o(\Omega) R_o(\Omega^\prime) \sum\limits_{i|d_i\geq\mathrm{thr}} p(\Omega|d_i) p(\Omega^\prime|d_i \right] \\
      & = \int d\Omega d\Omega^\prime \varepsilon(\Omega) \varepsilon(\Omega^\prime) P(\Omega, \Omega^\prime)
\end{align}
which implies that $P(\Omega, \Omega^\prime)$ is positive semi-definite and therefore $P^{-1} = \mathrm{cov}\left(\varepsilon(\Omega), \varepsilon(\Omega^\prime)\right)$ is also positive semi-definie.

As a purely pedagogical note, we examine the linear cofactor for $\varepsilon$
\begin{equation}\label{eq:linear cofactor}
    R_o \left( - \int dt \mathcal{N} p_\Omega + \sum\limits_{i|d_i\geq\mathrm{thr}} p(\Omega|d_i)\right)
\end{equation}
The first term is the Poisson factor accounting for how much exposure we have in any particular direction. 
All else being equal, if the exposure increases in a direction without more associated detections, the anisotropy is infered to be negative ($\varepsilon < 0$). 
Similarly, if there is an excess of detections infered from a certain direction without an increased exposure, the anisotropy is infered to be positive ($\varepsilon > 0$).
We also note that the intuitive notion of simply ``adding posteriors'' to look for excesses emerges naturally in the linear approximaiton, although it is too naive for the more general inference.

To infer the posterior process for $\varepsilon$, we can assume a Guassian Process (GP) prior for $\varepsilon$ with some covariance: $\varepsilon \sim \mathcal{N}\left(\mathrm{mean}=0;\ \mathrm{cov}=\Xi\right)$.
A common choice might be the squared-exponential covariance kernel such that
\begin{equation}
    \left. \text{cov}(\varepsilon, \varepsilon^\prime) = \sigma^2 \exp \left[ -\frac{\left(\Delta \theta\right)^2}{2l^2}\right] \quad  \right| \ \Delta \theta = \cos^{-1}\left(\cos\theta\cos\theta^\prime + \sin\theta\sin\theta^\prime\cos(\phi-\phi^\prime)\right)
\end{equation}
If we wish to avoid introducing a length scale in our prior (i.e.: we're interested in measureing clustering), we can take the limit $l \rightarrow 0$ to obtain a ``white noise'' prior instead.
However, maximizing the associated posterior as a function of $l$ may be a natural way to infer length scales in the data.

Assuming the GP prior and completing the square yields analytic inferences for $\epsilon$ at whatever resolution is desired along with error estimates.
Furthermore, because we assume $\varepsilon \ll 1 \ \forall \ \Omega$, zero-mean Gaussian priors on $\varepsilon$ are not unreasonable.
To wit, we obtain
\begin{equation}
    p(\varepsilon|\mathrm{data}, R_o) \sim \mathcal{N}\left( \mathrm{mean}=\int d\Omega^\prime \Gamma(\Omega, \Omega^\prime) A(\Omega^\prime);\ \mathrm{cov}=\Gamma^{-1}\right)
\end{equation}
where
\begin{align}
    A(\Omega) & = R_o\left( -\int dt\mathcal{N}p_\Omega + \sum\limits_{i|d_i\geq\mathrm{thr}} p(\Omega|d_i) \right) \\
    \Gamma(\Omega, \Omega^\prime) & = \Xi^{-1} + R_o(\Omega) R_o(\Omega^\prime) \sum\limits_{i|d_i\geq\mathrm{thr}} p(\Omega|d_i)p(\Omega^\prime|d_i)
\end{align}
and the posterior's normalization depends on $R_o$.

Furthermore, because $\varepsilon\ll1$, we should be able to infer $R_o$ independently of $\varepsilon$ and then simply marginalize over it here.
\begin{equation}
    p(\varepsilon|\mathrm{data}) = \int dR_o\, p(\varepsilon|\mathrm{data},R_o)p(R_o|\mathrm{data}) \propto p(\varepsilon) \int dR_o\, p(\mathrm{data}|R_o, \varepsilon)p(R_o)
\end{equation}
Alternatively, we can analytically marginalize over $\varepsilon$ to derive a likelihood for $R_o$ alone, from which we can sample $R_o$ and then use that empirical distribution to make inferences about $\varepsilon$.
\begin{equation}
    p(R_o|\mathrm{data}) \propto p(R_o) \int \mathcal{D}\varepsilon\, p(\mathrm{data}|R_o, \varepsilon)p(\varepsilon)
\end{equation}
This has the advantage of letting us quickly, and quasi-analytically, obtain posterior processes for $\varepsilon$ using already published estimates of $R_o$, exposure estimates, and posterior skymaps for individual detections alone. 

We also note that we can examine how informative the data is by investigating the covariance matrix for $\varepsilon$.
For instance, we note that the covariance matrix is inversely proportional to the number of detected events (roughly), meaning the side of confidence regions should scale as $\propto 1/\sqrt{N_\mathrm{events}}$.
This will immediately tell us whether our approximation $\varepsilon\ll1\ \forall\ \Omega$ can be justified by the likelihood, or whether the uncertainties are large enough that any constraints will come from the prior alone.

\subsubsection*{model complexity and null eigenvectors}

To determine how informative the data can be in our perturbative analysis, we examine the eigenvalues of $P(\Omega, \Omega^\prime) = R_o(\Omega) R_o(\Omega^\prime) \sum\limits_i p(\Omega|d_i) p(\Omega^\prime|d_i)$.
Indeed, this matrix appears to have an extremely large number of null eigenvectors. 
Although this has not been demonstrated rigorously, it seems like the number of non-zero eigenvalues is approximately equal to the number of events included in the sum.
This means that we cannot constrain arbitrarily complex models with finite amounds of data.
It also tells us which particular correlations between different $\varepsilon(\Omega)$ can be constrained with existing data; if one allows oneself to gerrymander ``model bins'' according to the non-null eigenvectors one can define something like the ``smallest possible data-informed model.''

Again, because all our inference is based on inner products, we can alternatively specify $\varepsilon_{lm}$ and a poserteriori distributions for sets of these coefficients. 
However, we expect that there will continue to be many null-eigenvectors in this representation as well.

We note that applying a non-trivial prior ($\Xi$), such as the squared-exponential Gaussian Process prior we suggested, can break these degeneracies and allow one to compute a posteriori processes for arbitrarily complicated $\varepsilon$.
However, the vast majority of ``measurements'' will be prior dominated in that case.

\subsubsection*{implications for CMB analysis}

This perturbative approach may be naturally applicable to CMB analyses looking for anisotropies. 
Indeed, it may provide an analytic estimate of the a posteriori joint distribution for the $a_{lm}$, which in turn can produce a posteriori joint distributions for the $C_{lm}$.
This is particularly attractive because the observed CMB signal is dominated by a monopole and dipole with much smaller additional anisotropies; this is exactly the case we consider with a ``known model'' $R_o$ and small perturbations $\varepsilon$.

However, before claiming a perfect match we should read instrument paper describing how the CMB detectors actually record data and what it means to have ``events'' in that context. 
Furthermore, the notion of $p(\Omega|d_i)$ will likely be somewhat complicated, although it may be reasonable to assume a fixed point-spread-function for each pixel.
Out likelihood also assumes that rates are low enough that the expected count above threshold in our detectors within some small time window is much less than 1.
Depending on how CMB detectors read out their data, this may be violated and we may need a more complicated likelihood that more accurately describes the detector response.

%-------------------------------------------------
\end{document}
